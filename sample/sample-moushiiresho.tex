\documentclass[a4paper,papersize,12pt]{jsarticle}
%\documentclass[a4paper,papersize,12pt,uplatex]{jsarticle}
\usepackage{kian}

% 発信人につき,
% 全体の字下げを24文字,肩書を2文字,
% 肩書と名前の間を1文字,名前を7文字にする
\senderhlength{24}{2}{1}{7}
% タイトルを次のとおり表示する
%  整理番号 → 日付 → 0.75行 → 標題(申入書) → 0.75行
%  → 受信人 → 0.5行 → 発信人
\maketitlecontents{id+++t+++r++s}

\begin{document}

% 整理番号は右寄せで,"発"は括弧で囲まない
\idcodes{[r]{児童会[発]345}}

\title{申入書}

% 年を"西暦(和暦)"で表示する
\両暦
\date{\today[0]}

\receivers{{%
  % 行間を25%だけ詰める
  X=\addtolength{\baselineskip}{-.25\baselineskip}\\
  赤白小学校//校長:校長/太郎:殿
}}

\senders{{
  % 行間を25%だけ詰める
  X=\addtolength{\baselineskip}{-.25\baselineskip}\\
  赤白小学校児童会//会長:会長/次郎\\
  (公印省略)
}}

\maketitle

\sectionZ{}

\sectionA{申入れの趣旨}

一輪車を購入していただきたい。

\sectionA{申入れの理由}

本校には一輪車がありませんが,
下記のとおり,一輪車は非常に優れた遊具です。

\hfil{記}

% セクションBを"a),b)…"で表示する。
\renewcommand{\printsectionB}[1]{\alph{sectionB})}

\sectionB*{}良い運動になります。

\sectionB*{}バランス感覚が養われます。

\sectionB*{}努力することが学べます。

…

\stateZ
このような優れた遊具を購入しない手はありません。

したがって,
一輪車を購入していただきたく,申し入れるしだいです。

% セクションBの番号を標準に戻す
\let\printsectionB\normalprintsectionB

\end{document}
