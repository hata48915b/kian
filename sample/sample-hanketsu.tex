\documentclass[a4paper,papersize,12pt]{jsarticle}
%\documentclass[a4paper,papersize,12pt,uplatex]{jsarticle}
\usepackage{kian}

% 当事者の住所は2文字だけ字下げする
\personaddresshlength{2}{}{}{}{}{}
% 当事者の肩書付き名前は10文字だけ字下げする
\personnamehlength{10}{}{}{}
% セクションZの上に間を空けない
\separateZ{0}
% セクションAの上に間を空けない
\separateA{0}
% セクションBの上に間を空けない
\separateB{0}
% 題目(判決,主文等)の幅は10文字
\subjecthlength{}{10}
% タイトルは事件番号と標題(判決)だけ
\maketitlecontents{it}

\begin{document}

\idcodes{
  {平成13年3月4日判決言渡:同日原本領収:裁判所書記官}
  [o]{H12(ワ)345:優勝旗引渡請求事件}
  {口頭弁論終結日:平成28年6月23日}
}

% タイトルは普通の文字サイズ
\title[n]{判決}

\maketitle

\person{
  a=赤白市赤白町赤白/8丁目9番1号/赤白小学校\\
  原告:運動会赤組\\同代表者応援団長:赤色/壱郎\\
  同訴訟代理人/弁護士:赤色/弐子\\同 :赤色/参郎\\
}

\person{
  a=赤白市赤白町赤白/8丁目9番1号/赤白小学校校長室\\
  被告:校長/太郎
}

\person{
  a=赤白市赤白町赤白/8丁目9番1号/赤白小学校\\
  被告:運動会白組\\同代表者応援団長:白色/壱郎
}

\sectionZ{主文}

% 全体の字下げを5文字にする
\downX{5}

\sectionB*{}被告らは,原告に対し,連帯して優勝旗を引き渡せ。

\sectionB*{}訴訟費用は,被告らの負担とする。

% 全体の字下げをデフォルトに戻す
\downX{d}

\sectionZ{事実及び理由}

\sectionA{請求}

\sectionB*{}被告らは,原告に対し,連帯して優勝旗を引き渡せ。

\sectionB*{}訴訟費用は,被告らの負担とする。

\sectionA{事案の概要}

\sectionB{事案の要旨}

本件は,原告が,
赤白小学校平成12年度運動会において優勝したとして,
優勝旗の引渡しを求めたものである。

\sectionB{前提事実}

\sectionC{当事者}

…

\sectionC{平成11年度の運動会}

…

\sectionC{平成12年度の運動会}

…

\sectionB{争点及びこれに対する当事者の主張}

…

\sectionA{当裁判所の判断}

…

% 空行を1行だけ入れる
\vspaceonkian{1}

\person{赤白小学校裁判所//裁判官:裁判/花子}

\end{document}
