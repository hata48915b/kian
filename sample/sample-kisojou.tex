\documentclass[a4paper,papersize,12pt]{jsarticle}
%\documentclass[a4paper,papersize,12pt,uplatex]{jsarticle}
\usepackage{kian}

% 用紙の上を広く使う
\narrowhead

\begin{document}

% 事件番号は右寄せ
\idcodes{[r]{H12[検]345}}

\title{起訴状}

\date{\today[0]}

\receivers{{:赤白小学校裁判所:殿}}

\senders{{赤白小学校検察庁//検察官検事::}}

% タイトルを次のとおり表示する
%  事件番号 → 0.75行 → 標題(起訴状) → 0.75行
%  → 日付 → 0.5行 → 裁判所 → 0.5行 → 検察官名
\maketitle[i+++t+++d++r++s]

\sectionZ{}

下記被告事件につき公訴を提起する。

\hfil{記}

% 肩書は左寄せ
\person[l]{
  D=赤白市赤白町赤白/8丁目9番1号\\
  H=赤白市赤白町赤白/8丁目9番1号/赤白小学校\\
  J=小学生\\
  N=在宅:赤色/壱郎\\
  B=平成元年1月1日
}

% 題目(公訴事実,罪名及び罰条)の幅を10文字にする
\subjecthlength{}{10}

\sectionZ{公訴事実}

被告人は,正当な理由がないのに,

\sectionA*{}平成12年3月4日午後10時11分頃,赤白小学校2階ろう下を走り,

\sectionA*{}平成12年3月5日午後12時13分頃,赤白小学校3階ろう下を走り,

\stateZ\noindent
通行人と衝突する危険を発生させたものである。

\sectionZ{罪名及び罰条}

% 箇条書きの項目の幅を20文字にする
\downY{20}

\itemonkian{赤白小学校校則違反}同規則10条

% 箇条書きの項目の幅をデフォルトに戻す
\downY{d}

\end{document}
